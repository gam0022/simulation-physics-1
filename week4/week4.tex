\documentclass[a4j,11pt]{jarticle}
\usepackage[dvipdfm]{graphicx}
\usepackage{lastpage}
\usepackage{fancyhdr}
\usepackage{listings, jlisting}
\renewcommand{\lstlistingname}{リスト}
\lstset{language=c,
    basicstyle=\ttfamily\scriptsize,
    commentstyle=\textit,
    classoffset=1,
    keywordstyle=\bfseries,                                                     
    frame=tRBl,                                                                  
    framesep=5pt,                                                              
    showstringspaces=false,                                                     
    numbers=left,                                                               
    stepnumber=1,                                                               
    numberstyle=\tiny,                                                          
    tabsize=2                                                                   
}


\makeatletter
\title{第4回}
\author{}
\date{2012年1月11日}

\pagestyle{fancy}

% headers & footers
\lhead{シミュレーション物理 \@title 提出日:\@date\\\@author}
\chead{}
\rhead{}
\lfoot{}
\cfoot{\thepage /\pageref{LastPage}}
\rfoot{}
\renewcommand{\headrulewidth}{0pt}
\renewcommand{\footrulewidth}{0pt}
\makeatother

\begin{document}
\section*{練習問題(7)}
\subsection*{引力なしの膨張シミュレーションプログラム}

以下に作成したプログラムex7.cのソースコードを載せる。
\begin{lstlisting}{caption=ex7.c,label=ラベル}
//-1から1の範囲で半径1の円内にある点の集合500組
#include<stdio.h>
#include<stdlib.h>
#define N 500

void makefile(double x0[],double y0[],double x[],double y[]){
    FILE *fo;
    char *fname;
    fname="data.csv";
    if((fo=fopen(fname,"w")) == NULL){
        printf("File[%s] does not open!!\n",fname);
        exit(0);
    }
    int i;
    for(i=0;i<N;i++)
        fprintf(fo,"%f,%f,%f,%f\n",x0[i],y0[i],x[i],y[i]);

    fclose(fo);
}


void makestar(double x0[],double y0[]){

    int count,n,i;    //countが200になるまで
    double x,y;
    n = 1000;       //countが200になるように
    count = 0;

    srand(193);     //設定
    for(i=0;i<=n;i++) {
        if(count < N){
            /*-1以上1以下の乱数を生成*/
            x = (2.0*rand()/(double)RAND_MAX)-1.0;  //castしておく必要あり
            y = (2.0*rand()/(double)RAND_MAX)-1.0;

            /*-1<=x,y<=1の範囲にある中心原点半径1の
              扇形の中に乱数による点が入ったらカウントする*/
            if((x*x+y*y)<=1.0) {
                x0[count]=x;
                y0[count]=y;
                count++;
            }
            if(i==n) printf("we cannot have suitable data\n");

        }else{break;} 
    }
}

void simulate(double x0[],double y0[],double H,int org,double dt,int nk){
    double vx[N],vy[N];
    int i,t;
    for(i = 0;i < N;i++){
        vx[i] = H * x0[i],vy[i] = H * y0[i];   //初速度を設定
        x0[i] += org,y0[i] += org;             //全天体の位置を第一象限へ  
    }
    for(t = 0;t <= nk;t++)
        for(i = 0;i < N;i++)
            x0[i] += vx[i]*dt,y0[i] += vy[i]*dt; 
}


int  main(){
    double x0[N],y0[N];
    double x[N],y[N];    
    makestar(x0,y0);
    int i;
    for(i = 0;i < N;i++)
        x[i]=x0[i],y[i]=y0[i];
    simulate(x,y,3.5,50,0.1,20);
    makefile(x0,y0,x,y);
    return 0;
}

\end{lstlisting}
計算は基本的に配列x0,y0で行うようにした。また、H,org,dt,nkが分かれば応用がきくようにした。
結果の値は初期値と膨張シミュレーション後の値を一つのdata.csvというファイルに書き込まれるように設定した。
図1に初期設定の散布図を図2に膨張シミュレーション計算後の散布図を示す。
\begin{figure}[htbp]
\begin{center}
\includegraphics[width=100mm]{first.eps}
\end{center}
\caption{初期設定の散布図}
\label{fig:one}
\end{figure}


\begin{figure}[htbp]
\begin{center}
\includegraphics[width=100mm]{last.eps}
\end{center}
\caption{膨張計算後の散布図}
\label{fig:one}
\end{figure}


\section*{練習問題(9)}
\subsection*{ガウスザイデル法を偏微分方程式に適応するシミュレーションプログラム}

以下に作成したプログラムGauss.javaのソースコードを載せる。
\begin{lstlisting}{caption=gauss.java,label=ラベル}
public class Gauss{
    public static void main(String[] args){
        int T = 2000;
        calgauss gauss1 = new calgauss(2);  //オブジェクトの作成、初期化
        for(int i = 0;i < T;i++) 
            gauss1.calculate(1.0,1);
        gauss1.printMatrix();
    }
}

class calgauss{
    private double[][] phi;
    private int ix,iy;
    private double p1,p2,rou;

    calgauss(int N){      //Nは計算したい行列の値
        phi = new double[N+2][N+2];
        for (int k = 0; k < N+2; k++) 
            for (int j = 0; j < N+2; j++) phi[k][j] = 0;
        phi[1][3]=22.5;
        phi[2][3]=36;
        phi[3][1]=-4.5;
        phi[3][2]=9;
    }

    public void calculate(double G,int dx){
        for(ix = 1;ix <= phi.length-2;ix++)
            for(iy = 1;iy <= phi.length-2;iy++){
                p1 = phi[ix+1][iy]+phi[ix-1][iy]+phi[ix][iy+1]+phi[ix][iy-1];
                rou = 6 * ix - 3 * iy;
                p2 =  G * rou * dx * dx;
                phi[ix][iy] = (p1 - p2)/4;
            }
    }

    public void printMatrix(){
        for(int i = phi.length - 2;i >= 1;i--){
            for(int j = 1;j < phi.length - 1;j++)
                System.out.printf("%7f " ,phi[j][i]);
            System.out.println();
        }
    }
}


\end{lstlisting}

以下に実行結果をのせる。結果は行列形式で表示した。
解析解の結果と一致している。
\begin{lstlisting}{caption=実行結果,label=ラベル}
juo:week4 jyuo$ java Gauss
9.000000 12.000000 
1.500000 0.000000 
\end{lstlisting}    
今回はroの計算式と表示行列の大きさ、境界の値が分かればいくらでも応用が利くようにした。
%\subsubsection*{(1-1)}

\end{document}
